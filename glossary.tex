\makeglossaries

\newglossaryentry{open-source}
{
	name=open-source,
	description={qualifie un logiciel dont le code initial est mis à disposition du grand public}
}

\newglossaryentry{DOM}
{
	name={Document Object Model},
	description={est l'arbre d'éléments qui compose un fichier HTML}
}

\newglossaryentry{NLP}
{
	name={Natural Language Processing},
	description={est une catégorie de techniques algorithmiques visant à comprendre un texte écrit dans une langue humaine}
}

\newglossaryentry{LSI}
{
	name={Latent Semantic Indexing},
	description={est le nom d'une catégorie de techniques algorithmiques visant à comprendre les relations entre des documents écrits dans une langue humaine}
}

\newglossaryentry{TF-IDF}
{
	name={Term Frequency-Inverse Document frequency},
	description={est le nom d'une technique de de keyword extraction}
}

\newglossaryentry{RAKE}
{
	name={Rapid Automatic Keyword Extraction},
	description={est le nom d'une technique de de keyword extraction}
}

\newglossaryentry{LDA}
{
	name={Latent Dirichelet Allocation},
	description={est le nom d'un modèle de topic modeling}
}

\newglossaryentry{stopword}
{
	name={stopword},
	description={est le nom donné aux mots qui servent généralement de liaison, que nous cherchons à ignorer lors d'analyses}
}

\newglossaryentry{endpoint}
{
	name={endpoint},
	description={est le nom donné à un point autonome accessible d'une API}
}

\newglossaryentry{API}
{
	name={Application Programming Interface},
	description={est un ensemble de méthodes invocables permettant l'accès à des fonctionnalités d'un logiciel}
}