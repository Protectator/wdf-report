%%%%%%%%%%%%%%%%%%%%%%%%%%%%%%
%                               %
% ----- CONCLUSION PROJET ----- %
%                               %
%%%%%%%%%%%%%%%%%%%%%%%%%%%%%%

\section{Conclusion du projet}

	\subsection{Délivrables}

		Ce projet est passé par plusieurs étapes distinctes qui ont mené à la production de plusieurs délivrables :
		\begin{itemize}
			\item Analyse des besoins
			\item Analyse des technologies
			\item Maquette papier
			\item Premier prototype : Développement et évaluation
			\item Deuxième prototype : Développement et évaluation
			\item Application finale
		\end{itemize}
		Chacune de ces étapes nous a amené à produire une itération supplémentaire contenant à la fois des nouveautés fonctionnelles, mais aussi des changements et améliorations ergonomiques. Chaque itération nous a donc rapproché du produit final.

	\subsection{Conclusion générale}

		Bien que le scope intial du projet soit surdimensionné, nous avons pu communiquer avec le client afin de clarifier quelles étaient les attentes principales. Les fonctionnalités centrales ont donc été implémentées, et le produit final répond donc aux besoins les plus importants du client. Ainsi, l'application finale permet de :
		\begin{itemize}
			\item Afficher des informations de l'ensemble des cas
			\item Rechercher des informations parmi les cas
			\item Afficher et modifier les données d'un cas particulier
			\item Afficher des informations de l'ensemble des patients
			\item Rechercher des informations parmi les patients
			\item Afficher et modifier les données d'un patient particulier
		\end{itemize}
		

	\subsection{Perspectives}

		Le projet dans son état final contient plusieurs pages non implémentées :

		\begin{itemize}
			\item Studies
			\item Stats
			\item Admin
		\end{itemize}

		Ces pages sont actuellement vides et un projet futur pourrait commencer par y ajouter du contenu, après avoir défini clairement leur utilité.

		De plus, l'application dans son ensemble pourrait bénéficier d'un véritable système de login, ainsi que d'une sécurité accrue, notamment dans les accès aux méthodes de l'API.

%%%%%%%%%%%%%%%%%%%%%%%%%%%%%%%%%%%
%                                    %
% ----- CONCLUSION PERSONNELLE ----- %
%                                    %
%%%%%%%%%%%%%%%%%%%%%%%%%%%%%%%%%%%

\section{Conclusion personnelle}

	J'ai beaucoup apprécié travailler sur ce projet pour deux raisons différentes :

	Tout d'abord, j'ai pu voir et gérer un processus centré sur l'utilisateur. Cette approche itérative change de ce que j'ai eu l'habitude de faire, et m'a permis de découvrir une autre manière de produire un résultat.

	Et également, j'ai pu me familiariser et utiliser dans une situation pratique certaines des dernières technologies du monde du Web. Bien que toutes les librairies ne m'aient pas plu (J'ai trouvé Material-UI assez lacunaire), j'ai beaucoup aimé travailler avec React et Flux, qui ont l'air assez matures pour pouvoir produire des interfaces de qualité et maintenables.
	