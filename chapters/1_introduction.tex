\section{Contexte}

	L'hôpital orthopédique du \gls{CHUV} utilise actuellement un fichier Excel partagé pour conserver les données médicales des patients. Bien que fonctionnelle, cette méthode est fastidieuse à la fois pour l'entrée de nouvelles données que pour la recherche parmi celles-ci, et ne possède pas de réel point positif autre que la facilité initiale de la mise en place.

	Le but de ce projet est de concevoir et développer une application Web multiplateforme pour la recherche et la visualisation de ces données médicales. Le projet comporte deux challenges techniques centraux :
	\begin{enumerate}
		\item L’architecture désirée nécessitera donc un développement sur toutes les couches du système, autrement dit « full stack ».
		\item L’aspect multiplateforme nécessitera l’utilisation d’outils et de librairies récentes.
	\end{enumerate}
	Ce projet est mené dans un but double : Il aura à la fois une utilité scientifique en tant que base de données pour des futures études, et une utilité clinique par l’utilisation de son interface par le personnel.

\section{Objectifs}

	À la fin du projet, nous serons en possession d’une application web multiplateforme permettant de :
	\begin{itemize}
		\item Rechercher et visualiser des données médicales
		\item Rechercher, visualiser et modifier des données de patients
		\item Insérer des données médicales
	\end{itemize}

	Un objectif est qualifié de secondaire, et sera réalisé en fonction de l’avancement du projet :
	\begin{itemize}
		\item Produire et montrer des statistiques sur les patients et les cas
	\end{itemize}

\section{Contraintes}

	Le développement du projet va se faire autour de quelques contraintes définies :
	\begin{itemize}
		\item L’application doit converser avec une base de données MySQL déjà existante.
		\item L’application doit être adaptée à une utilisation à la fois sur un appareil desktop, et sur un appareil mobile, ainsi que convenir à la plupart des navigateurs récents.
	\end{itemize}

\section{Méthodologie}

	Le développement se fera de manière itérative, afin de produire plusieurs maquettes et prototypes, avec des degrés de fidélité augmentant au fil du projet. 
	Le but est d'utiliser une méthodologie agile afin de pouvoir rapidement proposer des
	prototypes, et avoir des retours de clients assez tôt. On désire s'assurer que l'on produit une solution adaptée aux besoins du client.