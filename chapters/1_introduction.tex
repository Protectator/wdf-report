\section{Contexte}

	En janvier 2014, l'ONG Internet Society a publié le document Digital footprints qui aborde la question de la capacité que les webtrackers ont de définir le profile personnel des usagers d'Internet. [1]
	En 2016, Michal Kosinski chercheur à Stanford révèle les possibilités de définir un profile précis simplement en analysant les préférences (likes) enregistrées dans un profile Facebook. L'étude révèle que ce type d'analyse permet de mieux connaître une personne que ses proches et même de prévoir de probables comportements avec une grande précision. De plus, lors d'événements politiques majeurs ces techniques de profiling auraient été utilisées, comme dans le cadre des campagnes pour le Brexit ou pour l'élection du président américain Trump. [2]

	Bla contexte \gls{mot}.

	Bla liste
	\begin{enumerate}
		\item Truc 1
		\item Truc 2
	\end{enumerate}

\section{Objectifs}

	Le but de ce projet est de concevoir et d'implémenter un outil d'analyse de comportement d'utilisateurs d?applications Web pour révéler les potentiels de détection de profile des personnes (préférences, centre d'intérêt, orientations et opinions) en analysant les interactions et les informations échangées avec les applications Web. L'application développée dans ce projet a pour le but de sensibiliser le public et les médias à la question du profiling sur internet.

	Bla liste
	\begin{itemize}
		\item Truc 1
		\item Truc 2
	\end{itemize}

\section{Contraintes}

	Bla liste
	\begin{itemize}
		\item Truc 1
		\item Truc 2
	\end{itemize}

\section{Méthodologie}

	Bla